\documentclass[12pt]{article}

\usepackage[a4paper,margin=1.5cm]{geometry}
\usepackage{amsmath,amssymb,amsfonts}
\usepackage{physics}
\usepackage{chemformula}
\usepackage{graphicx}
\usepackage{xcolor}
\usepackage[colorlinks=true,allcolors=blue]{hyperref}
\usepackage[title,toc]{appendix}
\usepackage[style=phys,eprint=true]{biblatex}
\usepackage[nolist]{acronym}

\addbibresource{/home/corentin/Work/biblio-jabref/all-references.bib}

\newcommand\TODO[1]{\textcolor{red}{\textbf{[TODO: #1]}}}
\newcommand\Tc{T_{\mathcal C}}
\newcommand\intc{\int_{\mathcal C}}

\newcommand\up{\uparrow}
\newcommand\dn{\downarrow}
\newcommand\nodag{{\vphantom{\dagger}}}
%\DeclareMathOperator{\Arg}{Arg}

\begin{document}

\title{Solution of X-ray edge problem}
\author{Corentin Bertrand}

\maketitle

%\begin{abstract}
%    Abstract text
%\end{abstract}

%\tableofcontents


\begin{acronym}
	\acro{GF}{Green Function}
	\acro{QQMC}{Quantum Quasi-Monte Carlo}
	\acro{QD}{Quantum Dot}
\end{acronym}

\section{References}

The reference: \cite{NozieresDeDominicis1969}. Related: \cite{Roulet1969, NozieresRoulet1969}

Non equilibrium version: \cite{Aleiner1997}

\section{The problem}

\TODO{Generalize to larger impurities}

The X-ray edge problem is described by this Hamiltonian:
\begin{equation}
	H = H_c + H_d + V N_c \, \sum_{\sigma} d_\sigma^\dagger d_\sigma
\end{equation}
where
\begin{align}
	H_c ={}& \sum_n \qty( \varepsilon_n c_n^\dagger c_n + \gamma_n c_n^\dagger c_{n+1} + \text{h.c.})
	\\
	H_d ={}& E_d \sum_{\sigma} d_\sigma^\dagger d_\sigma + U d_\up^\dag d_\up^\nodag d_\dn^\dag d_\dn^\nodag
	\\
	N_c ={}& \sum_{n \in \mathcal{S}} v_n c_n^\dag c_n
\end{align}
$H_c$ is the Hamiltonian of an non-interacting system, typically a fermionic reservoir, which we represent as a 1D chain here. $H_d$ is the Hamiltonian of an isolated fermionic single orbital system, with Coulomb repulsion, which we will call \ac{QD}. These two systems are capacitively coupled through the last term. $N_c$ is the charge distribution of the reservoir that participates to the capacitive coupling.

As the quantum dot is isolated, its occupation is conserved. The X-ray edge problem can therefore be solved by dividing between its different occupation states. Hence we introduce the effective reservoir Hamiltonian for a quantum dot occupation $Q$:
\begin{equation}
	H_Q = H_c + V N_c Q
\end{equation}


We use a Keldysh formalism where $V$ is turned on at $t=0$ and the initial state is described by the density matrix
\begin{equation}
	\rho = \rho_c \; \rho_d
\end{equation}
with $\rho_c$ a non-equilibrium state of $H_c$ with partition function $Z_c$, and $\rho_d = e^{-\beta(H_d - \mu N_d)} / Z_d$ an equilibrium state of $H_d$ with partition function $Z_d = 1 + 2e^{-\beta(E_d - \mu)} + e^{-\beta(2(E_d - \mu) + U)}$. We will use $Z_0 = Z_c \; Z_d$.

We want the Green functions for $d_\sigma$ and for $c$.

\section{Green function for the \acf{QD}}

In the following we drop the spin index when irrelevant.

Let's work out the lesser \ac{GF}:
\begin{align}
	-i G^<_d(t, t') ={}& \expval{\tilde d_\up^\dagger(t') \tilde d_\up(t)}
	\\
	={}& \Tr \qty[\rho \; U^\dagger(t') e^{it'(E_d + U d^\dag_\dn d^\nodag_\dn)} d_\up^\dagger U(t') U^\dagger(t) d_\up e^{-it(E_d + U d^\dag_\dn d^\nodag_\dn)} U(t)]
	\\
	={}& \sum_{Q_\up, Q_\dn=0,1} \!\!\!\! P(Q_\up, Q_\dn) \Tr_c \qty[\rho_c \; \bra{Q_\up, Q_\dn} U^\dagger(t') e^{it'(E_d + U d^\dag_\dn d^\nodag_\dn)} d_\up^\dagger U(t') U^\dagger(t) d_\up e^{-it(E_d + U d^\dag_\dn d^\nodag_\dn)} U(t) \ket{Q_\up, Q_\dn}]
\end{align}
Here, we went in interaction picture, with the time evolution operator
\begin{gather}
	U(t) = \theta(t) T \exp{-i \int_0^t \dd{u} V N_c(u) d_\sigma^\dagger d_\sigma} + \theta(-t) [\text{adjoint}]
\end{gather}
with $Nc(u)$ in the interaction picture. We then expressed the trace over $d$ as a sum over its four states $\ket{Q_\up, Q_\dn}$, $Q_\sigma=0, 1$, and we used $d_\up(t) = e^{-it(E_d + U d^\dag_\dn d^\nodag_\dn)} d_\up$.
We introduced the probability $P(Q_\up, Q_\dn)$ to find the quantum dot in the state  $\ket{Q_\up, Q_\dn}$
\begin{equation}
	P(Q_\up, Q_\dn) = \frac{1}{Z_d} e^{-\beta[(E_d - \mu) (Q_\up + Q_\dn) + Q_\up Q_\dn U]}
\end{equation}

To continue we need to note that:
\begin{equation}
	U(t) \ket{Q_\up,Q_\dn} = \ket{Q_\up,Q_\dn} \tilde U_{Q_\up + Q_\dn}(t)
\end{equation}
with
\begin{equation}
	\tilde U_Q(t) = \theta(t) T \exp{-i \int_0^t \dd{u} V N_c(u) Q} + \theta(-t) [\text{adjoint}]
\end{equation}
and that $d_\up \ket{Q_\up,Q_\dn} = \delta_{Q_\up=1} \ket{0, Q_\dn}$.
Finally:
\begin{align}
	-i G^<_d(t, t') ={}& \sum_{Q_\dn=0,1} e^{i(t' - t) (E_d + U Q_\dn)} P(1, Q_\dn)
	\expval{ \tilde U_{1+Q_\dn}^\dagger(t') \tilde U_{Q_\dn}(t' - t) \tilde U_{1 + Q_\dn}(t)}_c
\end{align}
where $\expval{\ldots}_c = \Tr_c \qty[\rho_c \ldots ]$.

Similarly for the greater \ac{GF}:
\begin{align}
	i G^>_d(t, t') ={}& \expval{\tilde d_\up(t) \tilde d_\up^\dagger(t') }
	\\
	={}& \sum_{Q_\dn = 0,1} e^{i(t' - t) (E_d + U Q_\dn)} P(0, Q_\dn)
	\expval{ \tilde U_{Q_\dn}^\dag(t) \tilde U_{1 + Q_\dn}(t - t') \tilde U_{Q_\dn}(t')}_c
\end{align}

We define
\begin{gather}
	-i G^<_d(t, t') = \sum_{Q_\dn=0,1} e^{i(t' - t) (E_d + U Q_\dn)} P(1, Q_\dn) A^-_{Q_\dn + 1}(t, t')^*
	\\
	i G^>_d(t, t') = \sum_{Q_\dn=0,1} e^{i(t' - t) (E_d + U Q_\dn)} P(0, Q_\dn) A^+_{Q_\dn}(t, t')
\end{gather}
with
\begin{align}
%	A^{s}_{Q}(t, t') ={}& \expval{ \tilde U_{Q}^\dagger(t') \tilde U_{Q-1}(t' - t) \tilde U_{Q}(t)}_c
%	\\
	A^{\pm}_{Q}(t, t') ={}& \expval{ \tilde U_{Q}^\dag(t) \tilde U_{Q\pm1}(t - t') \tilde U_{Q}(t')}_c
\end{align}

We see that $A^{\pm}_Q(t', t) = A^{\pm}_Q(t, t')^*$.
Note that we only need $A_1^-, A_2^-, A_0^+, A_1^+$.


\subsection{Solving the $A$s}

In the steady state, we send the start of the contour to $-\infty$ and choose $t' = 0$, so that for all $t$:
\begin{align}
%	A^-_{Q}(t) ={}& \theta(t)\expval{ \tilde U_{Q-1}^\dag(t) \tilde U_{Q}(t)}_{H_{Q}} + \theta(-t)\expval{ \tilde U_{Q}(t) \tilde U_{Q-1}^\dag(t)}_{H_{Q}}
%	\\
	A^{\pm}_{Q}(t) ={}& \theta(t)\expval{ \tilde U_{Q}^\dag(t) \tilde U_{Q\pm1}(t)}_{H_{Q}} + \theta(-t)\expval{ \tilde U_{Q\pm1}(t) \tilde U_{Q}^\dag(t)}_{H_{Q}}
\end{align}

\subsubsection{Diagrammatic expression}

In the following we assume $t \ge 0$.
According to Wick theorem, $A^{\pm}_Q(t)$ is the sum of all closed (connected or disconnected) diagrams made of two-point vertices (labeled by a time within $[0, t]$, a site index $j \in \mathcal{S}$, and contributing $W v_j$ with $W=\pm V$) linked with $g^T(u, j; u', j') = -i\expval{T \tilde c_{j'}^\dag(u'^+) \tilde c_j(u)}_{H_Q}$, i.e. the time ordered Green functions solution of the reservoir with Hamiltonian $H_Q$. Each diagram is multiplied by $(-1)^{l}$ with $l$ the number of loops.

The linked-cluster theorem asserts that
\begin{equation}
	A^{\pm}_Q(t) = \exp(C(t))
\end{equation}
with $C(t)$ the sum of connected diagrams only (we omit $\pm$ and the dependence in $Q$ for simplicity). They can only be made of a single loop, so we can write explicitly
\begin{equation}
	C(t) = - \sum_{n \ge 1} \frac{W^n}{n} \int \dd{1} \ldots \dd{n} g^T(1; 2) \ldots g^T(n; 1)
\end{equation}
where we used the following notation shortcut:
\begin{equation}
	\int \dd{1} \rightarrow \sum_{j_1} v_{j_1} \int_0^t \dd{u_1}
\end{equation}
The $1/n$ factor stems for the choice of labeling.
Here it is useful to introduce the sum of all chain diagrams, or equivalently all connected diagrams with two external legs, with the following explicit expression\footnote{$\varphi_t$ is in fact the dressed time ordered Green function with a perturbation acting only between times $0$ and $t$.}:
\begin{equation}
	\varphi_t(u, j; u', j') = g^T(u, j; u', j') + \sum_{n \ge 1} W^n \int \dd{1} \ldots \dd{n} g^T(u, j; 1) g^T(1; 2) \ldots g^T(n; u', j')
\end{equation}

To relate $C$ and $\varphi$, we need to eliminate the previously mentioned factor $1/n$. 
This is done by taking the time derivative of $C$, which acts as fixing any of the vertex at time $t$, and therefore leads to%
\footnote{
Another possibility is to consider $W \pdv{C}{W}$ (keeping $g^T$ fixed), which has each diagram multiplied by its number of vertices. Then we see easily that
\begin{equation}
	W \pdv{C}{V} = W \int_0^t \dd{u} \sum_{j \in \mathcal{S}} v_j \; \varphi_t(u, j; u, j)
\end{equation}
}
\begin{equation}
	\dv{C}{t} = -W \sum_{j \in \mathcal{S}} v_j \; \varphi_t(t, j; t, j)
\end{equation}
remembering that equal time default ordering gives $\varphi_t(t, j; t, j) = \varphi_t(t, j; t^+, j)$.

Finally we need to obtain $\varphi$. 
$\varphi$ follows a (quasi\footnote{because vertices are limited to the time range $[0, t]$}) Dyson equation:
\begin{equation}
	\varphi_t(u, j; u', j') = g^T(u, j; u', j') + W \int \dd{1} g^T(u, j; 1) \varphi_t(1; u', j')
\end{equation}
which is enough to describe entirely $\varphi_t(\star; u', j')$ at fixed values of $u'$ and $j'$. As we only need the special case $u' = t$, this leads to $\# \mathcal{S}$ independent equations, parametrized by $n \in \mathcal{S}$:
\begin{equation}
	\label{eq:quasi_Dyson}
	\varphi_t(u, j; t, n) = g_{jn}^<(u - t) + W \sum_{k \in \mathcal{S}} v_k \qty[ \int_0^u \dd{v} g_{jk}^>(u - v) \varphi_t(v, k; t, n) + \int_u^t \dd{v} g_{jk}^<(u - v) \varphi_t(v, k; t, n) ]
\end{equation}
where $0 \le u \le t$ and $j \in \mathcal{S}$.
We did split $g^T$ into its greater and lesser components to make explicit the jump between the two, and used the fact we are in the steady-state.

In the special case where the capacitive coupling acts on a single site of the reservoir, e.g. $v_n = \delta_{n0}$, this simplifies into a single equation where all site indices can be dropped:
\begin{equation}
	\label{eq:quasi_Dyson_single_site}
	\varphi_t(u; t) = g^<(u - t) + W \int_0^u \dd{v} g^>(u - v) \varphi_t(v; t) + W \int_u^t \dd{v} g^<(u - v) \varphi_t(v; t)
\end{equation}


\subsubsection{Numerical solution}

We focus on the case where the \ac{QD} is coupled to a single site in the reservoir, i.e. $v_n = \delta_{n,0}$.
\TODO{Extend to general case}

We solve Eq.~\eqref{eq:quasi_Dyson} for each separate $t$ by using a representation of functions $[0, t] \rightarrow \mathbb{C}$ by their values on a grid $\{u_1, \ldots, u_N\}$. Each $u_i$ is associated to a basis function $\phi_i$ such that $\phi_i(u_j) = \delta_{ij}$. We represent $g^<$, $g^>$ and $f:u \rightarrow f(u) = \varphi_t(u, t)$ with this choice of grid and basis functions:
\begin{gather}
	f(u) = \sum_n f(u_n) \phi_n(u)
	\\
	g^>(u) = \sum_n g^>(u_n) \phi_n(u)
	\\
	g^<(u) = \sum_n g^<(u_n) \phi_n(u)
\end{gather}
In addition, we precompute, numerically or analytically, the (quasi) convolution integrals
\begin{gather}
	I^>_{km}(u_n) = \int_0^{u_n} \dd{v} \phi_k(u_n - v) \phi_m(v)
	\\
	I^<_{km}(u_n) = \int_{u_n}^t \dd{v} \phi_k(u_n - v) \phi_m(v)
\end{gather}

This leads to
\begin{equation}
	f(u_n) = g^<(u_n - t) + W \sum_{k,m} \qty[g^>(u_k) I^>_{km}(u_n) + g^<(u_k) I^<_{km}(u_n)] f(u_m)
\end{equation}
which can be seen as a linear algebra problem $M f = b$, using the shortcuts
\begin{align}
	f_n ={}& f(u_n)
	\\
	b_n ={}& g^<(u_n - t)
	\\
	M_{nm} ={}& \delta_{nm} - W \sum_k \qty[g^>(u_k) I^>_{km}(u_n) + g^<(u_k) I^<_{km}(u_n)]
\end{align}


\section{Green function in the reservoir}

\TODO{update with two spins}
\TODO{Update with generalized coupling}

We follow the same steps:
\begin{align}
	-i G^<_c(t, t') ={}& \expval{\tilde c^\dagger(t') \tilde c(t)}
	\\
	={}& \frac{1}{Z_0} \Tr_c \sum_{Q=0,1} \qty[e^{-\beta H_c} e^{-\beta E_0 Q} \tilde U_Q^\dagger(t') c^\dagger(t') \tilde U_Q(t') \tilde U_Q^\dagger(t) c(t) \tilde U_Q(t)]
	\\
	={}& -i \frac{1}{Z_b} \qty(g^<_c(t, t') + e^{-\beta E_0} \tilde G^<_c(t, t'))
\end{align}
And:
\begin{equation}
	-i G^>_c(t, t') = -i \frac{1}{Z_b} \qty(g^>_c(t, t') + e^{-\beta E_0} \tilde G^>_c(t, t'))
\end{equation}
$g_c$ are the \ac{GF} of the unperturbed problem $H_c + H_d$, while $\tilde G_c$ are the \ac{GF} of the effective problem for $Q=1$.



%\begin{figure}[t]
%    \centering
%    \includegraphics[width=8cm]{my_figure.pdf}
%    \caption{
%        \label{fig:the_figure}
%    }
%\end{figure}

%\begin{tabular}[t]{c|c}
%    $i$ & $n_i$ \\
%    \hline
%    0 & text \\
%    1 & other text
%\end{tabular}


%%%%%%%%%%%%%%%
%\noappendicestocpagenum
\begin{appendices}

\section{Green function of the disconnected reservoir: case of a quantum dot}

Here we describe the reservoir as a single site with onsite energy $\varepsilon$, coupled to two leads (left and right) with hopping energy $\gamma$. The leads are assumed featureless, i.e. flat density of state and infinite bandwidth, of the same temperature $1/\beta$, but with different chemical potentials $\mu_L$ and $\mu_R$. This describes a quantum dot.

We want the Green function $g$ of the central site.

The retarded GF is simply:
\begin{eqnarray}
	g^R(\omega) = \frac{1}{\omega - \varepsilon + i\Gamma}
\end{eqnarray}
with $\Gamma = 2\pi\gamma^2\rho$, and $\rho$ the (constant) density of state of each lead.

Inverting the Keldysh matrix, one find that the Keldysh component is
\begin{align}
	g^K(\omega) ={}& - g^R(\omega) \qty[g^{-1}]^K(\omega) g^A(\omega)
	\\
	={}& g^R(\omega) \qty[\Delta^K_L(\omega) + \Delta^K_R(\omega)] g^A(\omega)
\end{align}
with $\Delta^K_X(\omega) = i\Gamma \qty[2 n_F(\omega - \mu_X) - 1]$.
Now the time-ordered component reads
\begin{equation}
	2g^T(\omega) = g^R(\omega) \qty[\Delta^K_L(\omega) + \Delta^K_R(\omega) + 2(\omega - \varepsilon)] g^A(\omega)
\end{equation}

%Now let's go in the time domain.
%
%We need an approximation to do this analytically. Let's assume that the Lorentzian is wide enough to have negligible impact over the numerator of $g^T$. This means $\Gamma \gg |\varepsilon - \mu_X|$ and $1/\beta$. Then,
%\begin{equation}
%	g^T(\omega) \approx \frac{i}{\Gamma} \qty[n_F(\omega - \mu_R) + n_F(\omega - \mu_L) - 1] + \frac{\omega - \varepsilon}{\Gamma}
%\end{equation}
%Using the Fourier transform of the Fermi function $n_F(t) = \delta(t)/2 + i/2\beta \sinh(\pi t / \beta)$, we get
%\begin{equation}
%	g^T(t) \approx -\frac{e^{-i\mu_L t} + e^{-i\mu_R t}}{2\Gamma\beta \sinh(\pi t / \beta)}
%\end{equation}


This leads to the lesser and greater components:
\begin{align}
	2g^>(\omega) ={}& g^K(\omega) + g^R(\omega) - g^A(\omega)
	\\
	={}& 2i\Gamma \frac{n_F(\omega - \mu_R) + n_F(\omega - \mu_L) - 2}{(\omega - \varepsilon)^2 + \Gamma^2}
	\\
	2g^<(\omega) ={}& g^K(\omega) - g^R(\omega) + g^A(\omega)
	\\
	={}& 2i\Gamma \frac{n_F(\omega - \mu_R) + n_F(\omega - \mu_L)}{(\omega - \varepsilon)^2 + \Gamma^2}
\end{align}

Now let's go in the time domain.

We need an approximation to do this analytically. Let's assume that the Lorentzian is wide enough to have negligible impact over Fermi step functions. This means $\Gamma \gg |\varepsilon - \mu_X|$ and $1/\beta$. Then,
\begin{align}
	g^>(\omega) \approx{}& \frac{i}{\Gamma} \qty[n_F(\omega - \mu_R) + n_F(\omega - \mu_L) - 2]
	\\
	g^<(\omega) \approx{}& \frac{i}{\Gamma} \qty[n_F(\omega - \mu_R) + n_F(\omega - \mu_L)]
\end{align}
Using the Fourier transform of the Fermi function $n_F(t) = \delta(t)/2 + i/2\beta \sinh(\pi t / \beta)$, we get
\begin{align}
	g^>(t) \approx{}& \frac{i}{\Gamma} \qty[i\frac{e^{-i\mu_L t} + e^{-i\mu_R t}}{2\beta \sinh(\pi t / \beta)} - \delta(t)]
	\\
	g^<(t) \approx{}& \frac{i}{\Gamma} \qty[i\frac{e^{-i\mu_L t} + e^{-i\mu_R t}}{2\beta \sinh(\pi t / \beta)} + \delta(t)]
\end{align}
The time ordered Green function is then (not forgetting that equal time default order is equivalent to lesser):
\begin{equation}
	g^T(t) \approx - \frac{e^{-i\mu_L t} + e^{-i\mu_R t}}{2\Gamma\beta \sinh(\pi t / \beta)} + \frac{i}{\Gamma} \delta(t)
\end{equation}

%We now add a potential $QV$ on the central site, to get the Green function $g_Q$. TO do so we use the Dyson equation
%\begin{equation}
%	g_Q(\omega) = g(\omega) + g(\omega) QV g_Q(\omega)
%\end{equation}
%which gives
%\begin{align}
%	g_Q^R(\omega) ={}& \frac{1}{\omega - \varepsilon - VQ + i\Gamma}
%	\\
%	g_Q^K(\omega) ={}& g_Q^R(\omega) \qty[\Delta^K_L(\omega) + \Delta^K_R(\omega) + QV] g_Q^A(\omega)
%\end{align}
%Using our approximation for large $\Gamma$, we find
%\begin{equation}
%	g^{</>}_Q(\omega) = g^{</>}(\omega) + V / \Gamma^2
%\end{equation}

\end{appendices}

%\bibliography{yourbiblio.bib}
%\bibliographystyle{ieeetr}
\printbibliography

\end{document}
